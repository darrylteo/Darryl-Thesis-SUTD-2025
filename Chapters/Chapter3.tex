\chapter{Problem Description}
\label{chap3}
To whom do the benefits of FF-ICE R1 belongs to, and to what extent, would airspace operations benefit from incremental participation in the program? This section explores how we quantify the value of information sharing and collaboration for the implementation of FF-ICE R1, across various levels of participation among FIRs in the Association of Southeast Asian Nations (ASEAN) region, with the addition of the Sanya and Hong Kong region, which we recall as \textit{ASEAN Plus}. We will first extend the discussion from Chapter \ref{chap1}, of the processes through which information is shared and the systems supporting it, justify why a decentralized model is appropriate, and follow up with a description of the decentralized models applied to answer the key question --- \textit{What is the value of information sharing and collaboration?}. We then conclude the chapter with by listing the data sources that allow us to construct a realistic simulator of flights in the ASEAN Plus region that can answer the question.

We work towards answering the key question, through modeling the decentralized ATFM problem with a rolling horizon framework, where multiple flights are updated iteratively at each time step. Flight plans are fixed with a predetermined sequence of waypoints for each flight, making the primary challenge to ensure that the trajectories, defined as the Target Time Over (TTO) for each flight at each node along its flight plan, satisfy separation constraints between all flights at all nodes. At any given time step, flights may be either en route or queuing for departure: Two main types of decisions are considered: (i) Speed regulation: the FIR controller may direct an aircraft to speed up or slow down from its desired speed over each leg of the trajectory, within limits determined by the aircraft type and its phase of flight (ascent, cruise, or descent);  (ii) Delay management: the FIR controller may require a more significant delay than what can be achieved through speed regulation. In practice, this represents delaying a flight's departure, vectoring the aircraft to increase the distance to the next waypoint, or placing the aircraft in a holding pattern.

To address this problem, we develop a modeling framework consisting of two primary modules, which are further subdivided into individual algorithms and mechanisms (see Figure \ref{fig:framework}):

\begin{itemize}
    \item UPDATE module: Updates the simulator and prepares for the next timestep.
    \begin{itemize}
        \item Rolling Horizon Concept (RHC): Updates the current time window iteratively using a rolling time step (e.g., windows of 2 hours updated every 5 minutes). The RHC pulls in new active flights from a flight database, removes completed flights from the dataset, and reconciles flight paths. The reconciliation process is necessary because flights are updated independently in each FIR, potentially leading to inconsistent TTOs. This procedure is explained in Section \ref{sec:rolling}.
        \item Information Sharing: Updates flight information across FIRs for flights participating in the FF-ICE R1 information sharing program. The shared information may be used to enact collaborative ATFM initiatives such as ground delay programs and speed modulation.
    \end{itemize}
    \item SOLUTION module: Solves the problem of regulating flow at both airports and waypoints.
    \begin{itemize}
        \item Aircraft Flow Regulator (AFR): Assigns aircraft to channels, and schedules arrival and departure times for active flights nearing arrival or departure. Note that to make the approach easily generalizable, instead of modeling each airport's runway system (which can sometimes be a complex and unique system), the AFR adopts a more general framework. Each airport is modeled with three channels representing arrival-only, departure-only, and mixed operations. These channels are calibrated using capacity envelopes generated from historical data. The runway assignment procedure is explained in Section \ref{sec:AFR}.
        \item Waypoint Flow Regulator (WFR): Analyzes active flights and uses their initial planned TTOs, which may come from the previous time step (unless it is the flight’s first active cycle). The WFR optimizes the TTOs at subsequent waypoints, independently for each FIR, by adjusting the speed and holding times for each flight leg to satisfy separation requirements.
        \item Discrete Event Simulation (DES): Solves both the AFR and WFR simultaneously by utilizing an event queue to schedule flight legs sequentially. The DES will determine the entire flight trajectory, including the assignment of the arrival and departure channel, as well as the TTO at all nodes along the flight path.
    \end{itemize}
\end{itemize}

Once flights are solved by the SOLUTION module to satisfy separation requirements, the output data is then fed back to the UPDATE module for flight paths to be reconciled, the list of active flights to be updated, and collaborative information to be shared. This process is repeated until a user-defined termination event is reached, or all flights are completed. Note that in this problem, we do not consider flight level separations. Our viewpoint is therefore flow-based, limiting the number of aircraft traversing each waypoint per unit of time. We continue the chapter by providing a description for each of the submodules, with the RHC and information sharing submodules explained fully in this chapter. The AFR, WFR, and DES are more algorithmically complex, hence a separate chapter is devoted to each of the algorithms. 

\begin{figure}
    \centering
    \includegraphics[width=\textwidth]{Figures/framework model.pdf}
    \caption{Modeling Framework for the Integration of the UPDATE and SOLUTION Modules and Their Submodules}
    \label{fig:framework}
\end{figure}


%--------------------------------------------------------
\section{Rolling Horizon Concept}
\label{sec:rolling}
This section extends the Rolling Horizon Concept (RHC) introduced in Chapter \ref{chap2}, with a particular focus on its implementation in the FF-ICE simulator.

Due to inconsistencies arising from decentralized decisions made by independent FIRs, we have opted for the RHC approach to accurately model this operationally complex task, with the reconciliation of flight trajectories to ensure feasible flight plans at regular simulated time intervals. Decentralized decisions often result in optimized TTOs at various waypoints within each FIR, however, inconsistent intraflight TTOs may arise when considering prior decisions made by upstream FIRs. To resolve these discrepancies, the problem is modeled as a rolling horizon framework, where TTOs are iteratively updated and reconciled at each step, as new information becomes available. This approach aims to mimic how information is updated over time in real-world contexts, where both Air Traffic Control (ATC) and airlines may adapt their operations as they receive real-time information. Algorithm \ref{alg:RHC} outlines the steps of the RHC, using the AFR and WFR approach, while Figure \ref{fig:rhc} illustrates the underlying rationale and provides a simplified representation of how decentralized operations are modeled. The figure depicts two flights across two FIRs, both scheduled to arrive at FIR 1 simultaneously. Note however that in practice, the problem involves far more flights, FIRs, and time intervals, making it difficult to comprehensively capture in a single figure.

In this example, we assume that both aircraft have not yet entered FIR 1 or FIR 2, allowing all flight legs to be modified. The arrows in the lower half of each box (in green) represent the sequence of nodes traversed by Flight 1, while the arrows in the upper half of each box (in orange) represent the sequence of nodes traversed by Flight 2. There is only a single arc between any two nodes with minimum travel time of 10 minutes, and the minimum required separation for flights visiting the same node is 5 minutes.

At time step 1, which covers the time window from 00:00 to 02:00, the RHC starts by applying the AFR and WFR algorithms to independently optimize decisions for each FIR. Both nodes 3 and 4 have separation constraints violated, yet the decisions are made independently for each FIR since node 3 is managed by FIR 1 and node 4 is managed by FIR 2. In this example, the WFR delays Flight 2 in FIR 1 and Flight 1 in FIR 2, each by five minutes. The clock then advances (e.g., by five minutes), representing the time taken to update information across FIRs. When a new period begins, FIR 2 receives updated information that Flight 2 was delayed by FIR 1. As a result, Flight 2 is now expected to arrive at node 4 at 00:35 (00:30 + 5 minutes delay, considering the 10-minute minimum travel time from node 3 to node 4). This allows FIR 2 to re-optimize its previous decisions, adjusting Flight 1's timing back to 00:30 to avoid conflict with the updated timing of Flight 2.

It is important to note that in this simplified example with only two flights, FIR 2 can revise its decision for Flight 1 and restore its original schedule. However, in practice, the clock moves continuously, and other flights with updated timings may no longer be adjustable. These changes can introduce conflicts with flights already affected by earlier decisions, potentially preventing Flight 1 from returning to its original schedule. These dynamic updates highlight the decentralized nature of the problem, as independent decision-making can lead to situations where reversibility is no longer possible.

\begin{figure}
    \centering
    \includegraphics[width=\textwidth]{Figures/rhc new.pdf}
    \caption{Example of the Rolling Horizon Concept for Two Time Steps}
    \label{fig:rhc}
\end{figure}

\begin{algorithm}
\setstretch{1.3}
\caption{\texttt{RHC($\mathcal{F}_{db}$, $\mathcal{F}$, $\mathcal{F}_C$, n, t)}}
\label{alg:RHC}
\tcp{$\mathcal{F}_{db}$ represents the complete set of flights in the database}
\tcp{$\mathcal{F}$ represents the flights in current sliding window}
\tcp{n represents the number of time steps to run for}
\tcp{t represents the previous simulation time}
\tcp{$\Delta$ represents the time shift (5 mins in our example)}
\While{$\mathcal{F}\neq\emptyset$ $\text{AND n > 0}$} {
    \tcp{remove completed flights}
    \For{flight in $\mathcal{F}$}{
        \If{flight sufficiently far in the past} {
            $\mathcal{F}_C$ = $\mathcal{F}_C \cup flight$\\
            $\mathcal{F}$ = $\mathcal{F} - flight$
        }
    }
    \tcp{add new flights}
    \For{flight in $\mathcal{F}_{db}$} {
        \If{departure time > t + $\Delta$} {
            $\mathcal{F}$ = $\mathcal{F} \cup flight$
        }
    }
    \tcp{reconcile flights and share information}
    \For{flight in $\mathcal{F}$} {
        use forward recursion to reconcile TTOs for entire trajectory for $flight$\\
        \If{flight is information sharing} {
            use backward recursion to update backTTOs for entire trajectory for $flight$
        }
    }
    \tcp{designate channels for flights}
    \For{airport in airports} {
        run AFR on $airport$
    }
    \tcp{optimize each FIR, in a decentralized manner}
    \For{FIR in FIRs}{
        run WFR on $FIR$
    }

    $n = n - 1$\\
    $t = t + \Delta$
}
\end{algorithm}

In summary, the RHC works as such. The simulation advances by a fixed time interval, advancing the simulation clock and rolling horizon. Active flights which do not fall into the new rolling horizon are removed, and new flights that fall into the new rolling horizon are added to the list of active flights. We then reconcile flight legs, enforcing a minimum positive travel time between nodes. A transition from the current time step to the next, also involves updating the backTTO. This backTTO is conceptualized to assist a flight in adjusting its en-route and departure times. By setting the target time at each node in its flight path to meet the backTTO, delays at the destination airport can result in collaborative ground delays at the origin airport, in place of airborne holding, effectively reducing operational costs. Here, we define a generic function for forward recursion, as $TTO_i = TTO_{i-1}+d_i$, where the TTO at any node $i$ depends on the TTO of the previous node $i-1$ and a duration $d_i$ that is defined as the travel time from node $i-1$ to node $i$. Note that the forward recursion starts from the departure node, where $TTO_0 = departure \ time$. The backward recursion is similarly defined as $TTO_i = TTO_{i+1} - d_{i+1}$. Note that the backward recursion starts from the arrival node, where $TTO_N = arrival \ time$, and works backwards. Here, $N$ is defined as the index of the last node.

The updated trajectories, along with the new flights from the database, are then handed over to the SOLUTION module, which advances each aircraft along its flight path, transits aircraft control from one FIR to another, allowing each FIR to update the trajectories under its control. These decisions are in turn is handed back to the UPDATE module, looping until termination of the program. The implementation details, system definition charts, and user interface are given in Appendix \ref{AppendixC}.

Choosing an appropriate time shift for the RHC is the next step, and we thus refer to studies utilizing the RHC as mentioned in our literature review in Section \ref{sec:research}. The studies by \parencite{Henry2022} and \parencite{AbbaRapaya2021} have look ahead time shifts of 7 minutes and 30 minutes respectively. Furthermore, the seminal paper on ATFM \parencite{Balakrishnan2014} has also discussed the pros and cons of various time discretization values used in ATFM studies. Naturally, a shorter time shift better reduce any trajectory inconsistencies, however, the additional computational burden must be accounted for. One important consideration is that aircraft are not able to precisely follow a trajectory due to several factors such as the capabilities of the Flight Management System, weather and wind conditions, pilot behavior, etc. Moreover, from an operational standpoint, it is normal for ATCs to accommodate small deviations from the expected capacity using tactical maneuvers such as vectoring and speed changes \parencite{Balakrishnan2014}. Hence, this uncertainty in the ability of aircraft to conform to a precise 4D trajectory, coupled with the operational standard of ATCs, does provide leeway in allowing for a reasonably large time shift of a few minutes, with minimal impact on tactical flight operations. 

%--------------------------------------------------------
\section{Information Sharing}
\label{sec:IS}
As described in Chapter \ref{chap1}, the primary aim of this dissertation is to investigate the value of information sharing and collaboration. In particular, to provide a preliminary quantitative analysis of the FF-ICE R1 concept, through modeling and simulating air traffic in the ASEAN Plus region, under FF-ICE R0, R1, and mixed mode operations.

We define an information regime, as the nature of information each flight provides to each FIR. These include arrivals, departures, and en route flights. The information we are concerned with include the original flight plans, the flight plan updated by departure and en route changes, and delays due to congestion at the arrival airport. The information may be specified as three distinct definitions of Target Time Over (TTO):
\begin{itemize}
    \item The R0 TTO: the target time over a waypoint (or airport channel), under the original flight plan. This TTO, for all flights, is always available to all FIRs.
    \item The R1 TTO: the target time over a waypoint (or airport channel), updated by departure times and en route changes. This TTO, for both R0 and R1-level flights is available to any FIR which has current control of the aircraft. This TTO for R1-level flights are available to FIRs of which it passes through.
    \item The R1 BackTTO: the target time over a waypoint (or airport channel), back calculated from the desired arrival time at the destination airport. This TTO for R1-level flights is available to all FIRs through which it passes, and hidden for all R0-level flights.
\end{itemize}

Each flight through the region of interest is assumed to have a flight plan with a TTO assigned to each location on the plan. Initially, these TTO times would reflect an on-time departure and travel along each leg at the preferred speed for the designated equipment type and phase of flight. In the SOLUTION module, each FIR updates the planned TTOs for each flight, only for the locations within its boundaries as conditions and information change. In particular, under R0, information about a delayed departure is not communicated from the origin FIR to other FIR's visited on the flight plan. Consequently, in a subsequent FIR, if an aircraft fails to show up at a boundary entry location at its scheduled TTO, then that FIR will simply advance the TTO at that location to the current time (expecting the aircraft to arrive shortly) and adjust the other TTOs along the flight path within that FIR accordingly. Thus, in the absence of information sharing, the assembled flight plan, knitted together from the TTOs from different FIRs, can suffer major information inconsistencies. 

A feature of the R1 release for FF-ICE is that actual departure times would be shared among FIRs operating at release level R1. Consequently, each such FIR at R1 would update the planned TTOs for locations within its boundaries with the updated departure time. These updates happen at each timestep as part of the UPDATE module, updating TTOs and backTTOs for all flights participating in the information sharing program. For modelling simplicity, we extend this feature as follows. If a flight operates between two airports, both of which are in FIRs at release level R1, we refer to it as an \textit{R1-level flight}. For a R1-level flight, any update to departure time or planned leg travel time is used to update the TTO for all subsequent locations on the flight plan in a consistent manner. That is, even if the flight passes through an FIR at level R0, the TTOs in that region will be updated with upstream information. This clearly violates the concept of R0-level sharing, but we justify the assumption by  noting that most of the congestion in the network occurs at airports. Only a few waypoints experience congestion delays, and they are typically in the FIR of the origin or destination airport. Consequently, intermediate FIRs will experience little to no measurable benefit from such updated information. The alternative, that of modelling separate flight plans from the viewpoint of each FIR, is unnecessarily complex, and will not be considered in this study.

On the other hand, a major benefit of release level R1 is that it could facilitate collaborative ground-delay programs. To compute this, we work backwards from the planned arrival TTO for each R1-level flight and determine the TTO at each location which would enable the flight to achieve that arrival TTO using preferred flight durations on each subsequent leg. We refer to such a backward-calculated time as the backTTO, which is computed as follows: 
\begin{equation}
\label{eqn:bwrec1}
t_{fj-1}=t_{fj}-\widetilde{t}_{fj},
\end{equation}
for all $j=1,2,\cdots,|P_f| - 1$. Here, $t_{fj}$ denotes the target time over (TTO) at the $j^{th}$ node in path $P_f$ for flight $f$, and $\widetilde{t}_{fj}$ represents the preferred time of travel for this flight leg.

The backTTO at the origin airport for a flight is also known, in aviation terms, as the Calculated Take-Off Time (CTOT). If the flight departure could be delayed to the CTOT and nothing were to change en route, then the flight would not experience any airborne holding delays. In essence, the value of information sharing and collaboration would be displayed in the ability of the ground holding at the origin airport, to reduce the time a flight is airborne, while incurring as short a delay as possible at the destination airport. To further investigate the effectiveness of collaborative ground-delay programs, we later introduce a smoothing parameter, that adjusts the ground holding by a percentage of the CTOT.

%--------------------------------------------------------
\section{Airport Flow Regulator}
\label{sec:AFR}
We now transition to the submodules of the SOLUTION module. In the Airport Flow Regulator (AFR) problem, we consider a modification of the Aircraft Sequencing Problem (ASP) for multiple runways, in a decentralized rolling horizon setting, with multiple flights either queuing for arrival or departure at any given time. We seek to formulate an optimization problem to regulate the flow of aircraft at airports. 

To avoid confusion with the runway scheduling algorithms, we will recast the flow constraint for an airport in terms of three "channels": the arrival channel, the departure channel, and the common channel. Each channel has a fixed separation time chosen to satisfy the overall capacity envelope for the airport, where a capacity envelope specifies the maximum number of arrivals and departures at an airport, and the maximum number of operations if both arrivals and departures are present. The key idea behind this formulation is that if the flights satisfy the minimum required flight separation times, which are derived on the capacity envelopes, the overall flow of aircraft into and out of the airport will satisfy the capacity envelope. The derivation of these separation times are given in Section \ref{sec:envelope}.

The goal of the algorithm is to regulate flow at airports via choice of a preferred runway, and runway time for each arrival or departure aircraft. This would be done through two decisions. The first, is runway selection, where an arrival aircraft is assigned to either the arrival or common channel, and a departure aircraft, the departure or common channel. Note that these three channels may not always be available for scheduling at any given airport. However, there will always exist at least one available runway for arrival aircraft, and one available runway for departure aircraft. The second decision variable is imposing of ground delays for departure aircraft, and airborne holding for arrival aircraft. We choose not to prioritize one over the other a priori. However, delays would preferentially be assigned to departures in practice, due to the higher operational cost of airborne holding as compared to ground holding. A discussion of how such a modification may be implemented, is given in Section \ref{sec:QP}. 

Another statement of the AFR's objective is to match each flight to its desired arrival or departure time, while obeying all pairwise separation constraints. These desired times will differ based on the FF-ICE information systems (R0 or R1) for a given scenario. Each flight possesses an earliest and a desired runway time. These are static for R0 systems but dynamic for R1 systems. The desired runway time is a CTOT under R1 and can reflect destination airport (or en route) congestion. We imagine R0 systems reserve runway times for all flights based on the published flight plans, despite the uncertainty of arrivals.

Our interpretation of static earliest and desired runway times is determined under the assumption that flights are not privy to future delays that occur in other FIRs, hence the desired time of arrival and departure would follow that of the published flight plan for the FF-ICE R0 information regime. In contrast, for the FF-ICE R1 information regime, flights would be privy to future delays that occur in other FIRs. Hence, the desired time of arrival and departure would follow that of the backTTO, which takes into account the updated trajectory information and can be used to enact ground delays. The backTTO, which is updated through the information sharing submodule, is defined as to exist for flights, which in its flight plan, have both its arrival and departure airports in an FF-ICE R1 FIR. We stress again that both the AFR and WFR are interlinked and each algorithm's inputs and outputs are used to update the other. A sample interaction is portrayed in Figure \ref{fig:iterate}. The AFR is given authority over decision of runway choice and potential ground delays, as delineated by the double-lined box, and the WFR is given authority over TTOs along the entire trajectory, as delineated by the solid-lined box.
\begin{figure}[htbp]
    \centering
    \includegraphics[width = \textwidth]{Figures/afr to wfr bw.pdf}
    \caption{Iteration between the AFR and WFR}
    \label{fig:iterate}
\end{figure}

At the end of the AFR, each flight is assigned to one channel, either the departure or common channel at the origin airport, and one channel, either the arrival or common channel, at the destination airport. Once the arrival and departure times at any airport are determined by the AFR, these become the starting values for the WFR. For flights under the FF-ICE R0 regime, these arrival and departure times are computed based on the current TTOs, and are not enforced by the WFR. On the other hand, for flights under the FF-ICE R1 regime, these times are computed based on the backTTOs and are enforced by the WFR for departure flights, as a proxy for collaborative ground holding programs. An important point is that once a flight has been assigned to its channels at the origin and destination airport, these channels can be treated in the same way as waypoint nodes in the network. This style of modelling the channel assignment and ground delay programs, allow us to consider the airport flow regulator separately from the WFR, in Chapter \ref{chap5}.

Two algorithms have been applied to the AFR problem. They are a novel queue pressure algorithm and the classic First-Come-First-Served (FCFS) algorithm. The AFR, together with the applied algorithms and their results, on a regular and reduced capacity scenario for published flight schedules in the ASEAN Plus region, are further detailed in Chapter \ref{chap4}.

The AFR would play a critical role in displaying the value of information sharing and collaboration, where we hypothesize that with more information, the en route airborne holding may be replaced by ground holding at an airport, leading to a reduction in operational costs.

%--------------------------------------------------------
\section{Waypoint Flow Regulator}
\label{sec:WFR}
In the broader setting, we consider an ATFM model in a decentralized rolling horizon framework, with multiple flights either en route or queuing for departure at any given time. Assuming the flight plans are fixed in terms of the sequence of waypoints specified for each flight, the challenge at any given time is to ensure that the trajectories, which we define as the TTO of each flight at each node over the remainder of its flight plan, allow for adequate separation times between flights at each node. Here, a node refers to either an airport or waypoint in our flight network. We do not consider flight level separations or in-trail separations between waypoints. Hence, our viewpoint is flow-based, where a limit on the number of aircraft traversing each node per unit time is imposed. The problem setting is decentralized in the sense that each FIR is responsible for updating the trajectories of aircraft within its domain. We are led to such a formulation when formulating simulations of different information sharing regimes in decentralized air traffic management regions such as Southeast Asia. 
% After updating the trajectories across all FIRs, the simulation advances by a fixed time interval, advances each aircraft along its flight path, transits aircraft control from one FIR to another, advances the simulation clock and rolling horizon, and again allows each FIR to update the trajectories under its control. A transition from the current time step to the next, also involves updating the backTTO, defined similarly in Section \ref{sec:AFR}. This backTTO is conceptualized to assist a flight in adjusting its en-route and departure times. By setting the target time at each node in its flight path to meet the backTTO, delays at the destination airport can result in cooperative ground delays at the origin airport, in place of airborne holding, effectively reducing operational costs.

Within the control of a single FIR, we imagine two forms of planning control over each trajectory. The first is a speed regulation: we imagine the FIR controller can require each aircraft to speed up or slow down from its desired speed over each leg of the trajectory, within limits determined by the type of aircraft and its phase of flight (ascent, cruise, or descent). The computation of speed limits are presented in Appendix \ref{AppendixB}. For this dissertation, we allow speed changes throughout each phase of flight but recognize that a more realistic flow model would penalize each change. The second form of control is more extreme, it requires adding a delay to a trajectory leg of likely more than what could be accomplished with speed regulation. In practice, this could correspond to vectoring the aircraft to increase the distance to the next waypoint or putting the aircraft into a holding pattern. We refer to this second control as a hold command and penalize its use in the optimization objective more heavily than speed regulation. The penalty for holding commands can be made specific to the waypoint. In this way, the traffic simulation manager can steer the optimization algorithm to concentrate holding or vectoring actions to specific waypoints in the airspace.

Flight plans originate and terminate at airports, and we do not treat airport nodes differently from waypoint nodes. Recall that in the AFR overview in Section \ref{sec:AFR}, the AFR is liable for the channel selection at airports, and the initialized TTOs, which may include holding times, particularly for TTOs at the departure node. For the WFR, it is sufficient to recognize that a hold command applied to the first node of a trajectory would correspond to a request to employ a ground delay for the departing aircraft at the airport node. Since ground delays result in lower fuel consumption and lower ATC workload than airborne delays, we anticipate that the traffic manager will specify much lower penalties for holds at nodes which represent airports than at waypoint nodes.

Two algorithms have been applied to the WFR problem, Gradient Descent Ascent, and Simulated Annealing. The WFR, together with the applied algorithms and their results, on a regular and reduced capacity scenario for published flight schedules in the ASEAN Plus region, are further detailed in Chapter \ref{chap5}.

We posit that the WFR would be less critical than the AFR in showing the value of information sharing and collaboration, as the AFR has greater control of flight movement at the airport, including changing the runway and introducing ground holding. Nevertheless, the WFR can capture better resolution of where airborne holding takes place, and potentially the role of speed regulation. It could also capture the impact of waypoint constraints. 

%--------------------------------------------------------
\section{Discrete Event Simulation}
\label{sec:DES}
A Discrete Event Simulation (DES) abstracts out the optimization aspect and, in place, runs a rule-based simulation. The setting will match that of the AFR and WFR, with the goal of determining channels at airport nodes, and TTOs for all nodes, inclusive of both the waypoint and airport nodes. The DES will also utilize the TTOs and backTTOs in the same manner as the AFR/WFR model, with values are provided by the UPDATE module. We model the DES as an integrated flight scheduler, with the capability to both select channels at airports, and TTOs at all nodes, effectively performing the role of both the AFR and WFR. The results would not be directly comparable, between the AFR/WFR and the DES model, due to the airport channel and TTO choice being separated in the AFR/WFR model. Most importantly and central to this dissertation, the DES yields promising results that are useful in distilling the value of information sharing and collaboration, particularly from the use of collaborative ground delays. The DES also provides a realistic operational scenario from the perspective of ATCs. For ATC operations under regular conditions, decisions are typically rule-based and straightforward \parencite{Ma2024, Chandra2025}. Deviation from a fixed schedule typically occurs in the event of inclement weather or emergencies such a request for priority landing because a passenger onboard requires immediate medical attention. Since such events are rare and overly complex to model, they would not be considered in this simulation. Furthermore, a comparison of the AFR/WFR against the DES model provides insight on how a separation of concerns, between the airport channel and TTO selection, may affect delays.

In the main loop of the DES, we process all events in non-decreasing order of earliest available time. We then employ a First-Scheduled-First-Served (FSFS) algorithm, where in contrast to FCFS, flight events are processed in order of their scheduled time, rather than the time of which they arrive at a given node. This scheduling strategy is applied exclusively to R1-level flights. The FSFS is not new. We have found it similarly applied in other studies including \parencite{Ball2001, Bertsimas2011, Wambsganss2001}. We believe a FSFS policy conducted in tandem with ground delays are necessary because no flight would appreciate having ground delays imposed on it. However, if there is a benefit of node access priority, which reduces airborne delays or holding time, airspace users are then incentivized to adopt the collaborative policy.

Events may refer to either flight events or resource events, where a resource refers to a unique node-channel pair. Flight events first assign a TTO to the current flight leg that is being processed, subsequently adds a resource event to mark when the node will next be available, and finally adds an event to mark the next leg in its flight path as pending, if any. Resource events signify times at which a resource is available. An event indicating a specific resource is available, will trigger a check to assign the next available flight to this resource, if any. At the end of processing all events, the DES will have generated conflict free TTOs for all flights within the current simulation step, under the FSFS logic for R1-level flights, and FCFS logic otherwise.

%--------------------------------------------------------
\section{Data Sources}
\label{sec:datasources}
Our flight network, consisting of the nodes and arcs in the ASEAN Plus region, are derived from the Aeronautical Information Publication (AIP) documents published by each countries' civil aviation authority. Next, we determined the FIR boundaries, using geographical data from ICAO. The flight schedules were obtained from OAG and contain historical information on the origin and destination for each aircraft, together with the scheduled arrival and departure times, for a period in late October 2023. Using the nodes, arcs, and the origin and destination airports, we knitted together flight paths for all aircraft, using Dijkstra's shortest path algorithm. We provide a plot of the network of nodes and arcs, together with FIR boundaries, in Figures \ref{fig:nodes} and \ref{fig:arcs}, later in Chapter \ref{chap5}. Finally, flight profiles for each equipment type were obtained from the EUROCONTROL Aircraft Performance Database \parencite{EUROCONTROL2024}. The dataset used for our experimental runs are published flight plans for flights arriving to, or departing from, the ASEAN Plus region for seven days starting from 1 Oct 2023.