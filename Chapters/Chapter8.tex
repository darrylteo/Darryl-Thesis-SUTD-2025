\chapter{Conclusion}
\label{chap8}
Amidst factors including the continued strong growth in the aviation sector, the mismatch between airspace capacity and demand, and incremental implementation of the FF-ICE initiative, airspace management retains its position at the forefront of aviation research. Imbalance of capacity and demand is well known to lead to increased flight delays and high costs incurred by airlines, airports and passengers. As we recall, a study done by the Federal Aviation Administration (FAA) estimated delays to cost a total of 33.0 billion USD in 2019, with passengers bearing the brunt of the cost due to time lost from schedule buffers, delayed flights, flight cancellations, and missed connections \parencite{FAA2019}. The cost and time required to upgrade physical infrastructure to increase airspace capacity are substantial. Consequently, improving the efficiency of air traffic management has been regarded as a more practical focus, particularly within the research community. In the literature reviewed, there have been numerous studies on the airspace optimization, however, almost all the studies for complete trajectories assumed the presence of a central controller, such as the Federal Aviation Administration's Air Traffic Organization, that manages decisions. In a decentralized system like Southeast Asia, such control does not currently exist, resulting in tactical measures being implemented only within individual Flight Information Regions (FIRs). Therein lies the dichotomy, that most research is not applicable to decentralized airspace networks such as Southeast Asia, motivating the key question that this dissertation aims to address. Furthermore, while a number of papers and technical reports have been published on the operational concepts and qualitative analyses of FF-ICE \parencite{ICAO2012, Liang2021, EUROCONTROL2024}, there has been a lack of quantitative research on this topic. Therefore, the primary focus of this dissertation has been to address the gap in the literature, conducting a quantitative analysis on the benefits of information sharing for various combinations of participating stakeholders, using historical flight schedules and network capacity within the ASEAN region to simulate the ASEAN Plus airspace network. The results are promising, indicating airborne savings for information sharing, both between and within countries, highlighting the benefits of collaboration as the air travel industry continues to grow.

The software deliverable for this dissertation, and associated research project, the FF-ICE simulator, incorporates foundational algorithms and components that enable a systematic analysis of various scenarios within the ASEAN Plus airspace. With further data collection, the simulator could be expanded to model additional regions. Moreover, the FF-ICE simulator may be extended to support analyses such as re-routing strategies, alternative information-sharing regimes, and the evaluation of different concepts of operations through representative simulation scenarios. We believe this tool will assist ANSPs at all planning phases, from the strategic, to pre-tactical and tactical phase, and allow ANSPs to test and better understand the potential benefits of incremental participation in the FF-ICE initiative in a decentralized airspace network. The system definition and user interface design can be found in Appendix \ref{AppendixC}. The data collection, ideation, system architecture and programming of the FF-ICE simulator is a team effort by the researchers at the ASI, consisting of Prof. Peter Jackson, Prof. Nuno Ribeiro, Rakesh Nandi, Dai Gengling, and myself. Additionally, we thank Prof. Daniel Delahaye for providing constructive feedback on the operational standards of air traffic management, and code for the simulated annealing algorithm. 

%--------------------------------------------------------
\section{Summary of Research}
In this dissertation, we have proposed several frameworks for building a realistic flight simulator in decentralized airspace networks, such as those in Southeast Asia. First, we provided a brief overview of these methods in Chapter \ref{chap3}. The FF-ICE simulator consists of two modules, the UPDATE and SOLUTION module. The UPDATE module updates the simulator according to the selected information sharing regimes, and prepares for the next time step. The update is done through a Rolling Horizon Concept and Information Sharing between flights or regions participating in the FF-ICE R1 initiative. Next, the SOLUTION module solves the problem of regulating flow at both airports and waypoints, either using the Airport Flow Regulator (AFR) in conjunction with the Waypoint Flow Regulator (WFR) algorithm, or the Discrete Event Simulation (DES), which have demonstrated the ability to generate conflict-free trajectories. The UPDATE and SOLUTION modules operate alternately, exchanging data between each other at regular simulated time intervals to accurately represent operational flight activity, and uncertainty such as changes in the flight trajectories within the airspace network. This is achieved by simulating the transmission of information between aircraft and ANSPs, and updating of flight trajectories once per time step, which represents the sharing of information between participants of the FF-ICE R1 initiative. At the end of Chapter \ref{chap3}, We listed the data sources used to build the FF-ICE simulator.

In Chapter \ref{chap4}, we provide the mathematical formulation of the AFR, whose role is the scheduling of aircraft to channels in the context of ATFM measures and capacity envelopes. The capacity envelope concept, allowed us to model aircraft flow at airports using three channels, the arrival, departure, and common channel, with specified maximum flow rate at each channel. This flow rate was derived from the capacity envelope representing the maximum number of arrivals, departures, and total operations in a given time period of one hour, at all airports. We then described the mathematical formulation of converting these flow rates into minimum required separation times between aircraft, and applied a novel queue pressure algorithm, inspired by the shifting bottle procedure \parencite{Adams1988}. This algorithm, as the name implies, identifies the queue, either arrival or departure, as a bottleneck and attempts to first schedule flights from that queue. The results of the queue pressure algorithm were then compared against a classic FCFS algorithm. The results indicate no significant benefits, with the solution quality differing by less than 1\%. We believe this may be attributed to the ATFM model assumptions, where all flights at any given channel are required to satisfy a unique minimum separation requirement, rather than different separation requirements based on weight class, thereby lacking the benefits of flight sequencing. As such, we have opted to use the FCFS algorithm for the reasons of interpretability and quicker computation times.

In Chapter \ref{chap5}, we discuss the WFR. The goal of the WFR is to issue speed regulation and hold commands to aircraft such that the minimum required separation between any pair of aircraft at all nodes are observed. We presented the mathematical model and necessary assumptions, and proved propositions that guided our design choices, including how we handled frozen flight legs to reduce the state space of the problem. We highlighted that the WFR is a non-trivial combinatorial problem, which motivated our decision in applying heuristic methods. We then continued the chapter by describing the first algorithm applied to the WFR, the Gradient Descent Ascent (GDA) algorithm, including the formulation of the Lagrange dual and partial derivatives necessary for the computations. We also described some of the problems faced by the GDA algorithm, and how we overcame them. In particular, we formulated a pre-sequencing algorithm that provided a good initial solution to the GDA, which resulted in conflict-free trajectories. We conducted a similar study on the Simulated Annealing (SA) algorithm, and benchmarked the results of the GDA and SA against exact methods. Considering the reduced capacity case, GDA offered the best computational performance, resolving all conflicts within a decision window in under 9 seconds. By comparison, SA, EM with a 10-second time limit per region (EM10), and EM with a 600-second time limit per region (EM600), required 75, 102, and 5413 seconds, respectively. While all conflicts were resolved much more rapidly, the deviation from scheduled times under GDA was 26\% worse than the best solution obtained. The SA algorithm generated solutions that were 10\% better than GDA, 8\% better than EM10, but 13\% worse than EM600. We then weighed the benefits and drawbacks for each of the proposed methods. Finally, we concluded that the successful application of the GDA, SA, and exact algorithms to the WFR demonstrated the utility of the methodological framework developed in the chapter for simulating air traffic within the decentralized ASEAN Plus airspace network.

In Chapter \ref{chap6}, we present a rule-based modeling approach that addresses the problem in a conventional and deterministic manner. The Discrete Event Simulation (DES) framework encompasses the functionalities of both the AFR and the WFR, handling the assignment of airport channels to aircraft and the scheduling of conflict-free trajectories. In essence, the DES sequentially schedules available aircraft legs while considering applicable information-sharing regimes and enforcing minimum separation requirements. Flights participating in information sharing may substitute airborne holding with ground holding and are granted priority at downstream nodes under a FSFS policy. In contrast, flights that do not participate in information sharing are scheduled according to FCFS. The explicit FSFS mechanism for R1-level flights ensures that flights delayed due to CTOT restrictions are prioritized over flights that would not have been ahead without the CTOT delay. We argue that FSFS offers a more realistic representation of collaborative GDPs than FCFS. It is unlikely that airlines and origin airports would be willing to cooperate in a system from which they derive no operational benefit, one in which an aircraft faces departure disruptions and incur arrival delays without queue priority at the network nodes.

In Chapter \ref{chap7}, we argued why the DES is the algorithm of choice for the FF-ICE simulator SOLUTION module. We demonstrated that regional collaborative GDPs must incorporate some form of preferential treatment for delayed flights, which the results show the FSFS principles to be a suitable choice. With the coordination of GDPs and FSFS scheduling, the results demonstrate airborne savings of 12.3\% and 17.4\%, for flights with more than 5 minutes of airborne savings under the regular and reduced capacity case respectively. We also looked at different information regimes and demonstrated that the benefits increase with the number of airports participating and the congestion levels they face. Our findings show that a 20\% reduction in airspace capacity led to an increase of airborne savings of 200\% under the FF-ICE R1 regime. Importantly, given that incremental participation is anticipated to occur, we have also demonstrated that a single country, or a group of countries can benefit from such investment, without complete participation with the ASEAN Plus region. In particular, with all 14 FIRs participating as the baseline, the savings range from 8\% for a single FIR (Vietnam only), 22\% for four FIRs, to 46\% for six FIRs. We go further to measure the benefits a single airport pair collaboration can yield, demonstrating that flights participating in collaborative GDPs consistently enjoy reduced airborne delay. However, these benefits may be outweighed by the unpredictable rescheduling of other non-participating flights, particularly under time periods with tight capacity where any single rescheduled flight may cause downstream flights to be erratically rescheduled.

%--------------------------------------------------------
\section{Future Research Directions}
This dissertation has demonstrated the value of information sharing and collaboration as constant progress is made towards FF-ICE R1, and laid the groundwork for future research related to incremental participation in information sharing systems between regions, airports, and airlines. We believe the core contribution of our research, the FF-ICE simulator, and the insights derived from our analyses, will provide a foundation for future research to be conducted in the ASEAN airspace and beyond, particularly in response to growing capacity pressures. In light of the the constant pressure to address the ever growing demand for air travel and transport, we pose a list of questions and problems we would seek to answer over the next few years.

\begin{itemize}
    \item \textbf{Migration Towards FF-ICE R2:} FF-ICE R1 introduces services that allow for collaboration between flights at the pre-departure phase. Our research has demonstrated promising results for cooperation between FIRs, with the primary ATFM procedure selected as ground delay programs. As the global aviation agencies and regulators prepare for the next phase of FF-ICE, the FF-ICE R2, we plan to extend our research to incorporate and support the FF-ICE R2. To do so, we plan to continuously update our FF-ICE simulator to allow for post-departure negotiation between airspace users and ATM service providers. Possible updates include the possibility of re-routing flights, assigning the FF-ICE release level based on flights rather than region or airport, and an increased focus on the intersection between the fields of ATFM and TBO.
    \item \textbf{Testing the FF-ICE Simulator by Industry Stakeholders:} The FF-ICE simulator developed during our research has been handed over to and is currently under evaluation by the CAAS for further study through the simulation of various collaborative regimes. Their feedback will lead to further refinements of the capabilities of the FF-ICE simulator. Furthermore, if the results of the simulator can be validated through operational tests with other ANSPs in the ASEAN region, this would be the most accurate barometer of the accuracy of our modeling and assumptions of the FF-ICE concept. We would continue to work with CAAS and disseminate our work to other ANSPs in the region, facilitating the discussion and proposals of the collaborative regimes and ATFM measures that may be considered in future FF-ICE releases.
    \item \textbf{Compare the Results of Decentralized Operations to Centralized Operations:} The core theme of the dissertation has been studying decentralized operations, with results demonstrating that incremental participation in FF-ICE R1 will lead to a reduction in airborne time, as well as fuel consumption. As many studies on a single centralized ATC using exact methods have been published, such as in \parencite{Balakrishnan2014, Tan2021, Bolic2017}, future work may include collaborating with these authors to establish the differences in trajectories, airborne and fuel savings on a common dataset. An optimistic estimate of when all countries would collaborate would be decades at earliest due to sensitive data sharing and system upgrading costs. However, it could turn out that it is unnecessary in the first place, and partial collaboration could elicit most of the benefits. Such comparisons would allow us to establish a lower bound to the airspace optimization, gain deeper insight as to where inefficiencies arise from decentralized operations, and quantify the proportion of the benefits that can be gained from partial participation in collaborative decision-making programs in a decentralized airspace.
    \item \textbf{Propose Data-Backed Concept of Operations:} We aim to extend the research on FF-ICE, using our FF-ICE simulator to test and propose other concepts of operations, or ATFM procedures. Preliminary testing has been done on fuel consumption, and delaying transcontinental flights en route if delays are predicted to occur in the TMA. However, the results have been mixed, due in part to the sensitivity of fuel data collection efforts. Other operations that we could run simulations on include platooning and rule-based scheduling with different priority rules. Additionally, examining operations within the TMA could also uncover opportunities for collaboration and benefits could arise from employing different ways to delay flights such as vectoring, holding, and point merge, in a decentralized airspace network.
    \item \textbf{Increased Fidelity of the Model:} Our ATFM model is flow based, and does not take into account aircraft weight classes. In practice, weight classes determine the wake vortex turbulence and by extension, the required separation between aircraft on runways, and therefore sequencing of aircraft on runways by weight class. Hence, one potential avenue to enhance the fidelity of the FF-ICE simulator is to incorporate aircraft weight classes, which may enable more refined operational strategies and unlock new opportunities for optimizing the airspace network. We anticipate this change would yield larger benefits as studies have demonstrated a reduction in makespan of 5\% to 16\% \parencite{Balakrishnan2010, Desai2022}, in contrast to our study in Chapter \ref{chap4}, where a fixed separation time across all aircraft weight classes which was shown to be no better under a queue pressure or FCFS algorithm. The second way would be to limit the number of CTOT ground delays permitted per unit time. This reflects the operational procedures taken in Europe, where the number of CTOT slots are limited to prevent a concentration of aircraft within any specified time period \parencite{Ma2019}, for the purpose of more efficient airspace utilization.
\end{itemize}